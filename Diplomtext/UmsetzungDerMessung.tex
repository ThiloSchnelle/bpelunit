 Ziel: m�glichst wenige Eingriffe in die Logik des Programms.
  Damit die BPEL-Datei syntaktisch korrekt bleibt, muss die Datei f�r die Instrumentierung vorbereitet werden.

  \subsection{Statementabdeckung}

  
  Alle strukturierenden Aktivit�ten (bis auf Sequence) sind so konstruiert, dass sie in ihren Komponenten genau oder maximal eine Aktivit�t erlauben. Durch das Schachtelungsprinzip k�nnen trotzdem beliebig komplexe Strukturen an diesen Stellen eingef�gt werden. Wenn an diesen Stellen nur ein Basic-Aktivit�t vorhanden ist, dann kann man nicht ohne Weiteres dahinter oder davor eine zus�tzliche Aktivit�t hinzuf�gen. Das w�rde zur Verletzung der Syntaxregeln f�hren. Also muss man vor der Instrumentierung eine umschlie�ende strukturierende Aktivit�t hinzuf�gen. Da das Logging entweder direkt davor oder danach statfinden soll, ist die Sequence-Aktivit�t gut f�r diesen Zweck geeignet. 
  
  Innerhalb der Flow-Aktivit�ten erm�glicht das Link-Konzept die Synchronisationsabh�ngigkeiten zwischen den Aktivit�tten zu definieren und durch die darauf aufbauenden Bedingungen die Ausf�hrung der Aktivit�ten zu steuern. Durch das Einf�gen der Protokollieranweisungen alleine kann eine Basic-Aktivit�t, die das Ziel eines oder meheren Links ist, nicht protokolliert werden. ...
  
  L�sung Sequence-Element um diese Aktivit�t und den Link auf die Sequence verschieben. Damit ist die richtige Protokollierung unabh�ngig vom suppressJoin-Wertes garantiert. Die Crossing-Boundary Bedingung wird durch den Eingriff auch nicht verletzt.
  
  
 
\lstset{emph={joinCondition,target }, emphstyle=\color{blue}}
      \begin{lstlisting}[caption=Beispielcode]{Name}
<flow>
  <links>
    <link name="CtoD"/>
  </links>
  <receive name="C" ...>
    <source linkName="CtoD"/>
  </receive>
  <invoke ... joinCondition=...>
    <target linkName="CtoD"/>
  </invoke>
</flow>      \end{lstlisting}

\lstset{emph={[2]sequence}, emphstyle=[2]\color{red}}
      \begin{lstlisting}[caption=Beispielcode][firstnumber=1]{Name}
<flow>
  <links>
    <link name="CtoD"/>
  </links>
  <receive name="C" ...>
    <source linkName="CtoD"/>
  </receive>
  <sequence joinCondition=...>
    <target linkName="CtoD"/>
    <!--Logging-->
    <invoke .../>
    <!--Logging-->
  </sequence>
</flow>      \end{lstlisting}


\subsection{Zweigabdeckung}

Jede Kante wird geloggt durch Einf�gen von Marken vor den Aktivit�ten

otherwise oder else einf�gen

sequenzen Einf�gen

 Flow: beim Fehler.
 
 \subsection{Fault Handler Abdeckung}

\subsection{Compensation Handler Abdeckung}