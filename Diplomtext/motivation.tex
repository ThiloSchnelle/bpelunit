Wer benutz BPElUnit Was hat er davon

Wozu metriken f�r wenn


In einem Test-Prozess sind
Entwickler und Tester bestrebt m�glichst alle Teile des Quellcodes durch die Tests
mindestens einmal auszuf�hren. Ein JUnit-Test liefert jedoch als Ergebnis im wesentlichen
nur, ob er gelungen oder misslungen ist. �ber die Qualit�t eines Tests liefert
JUnit damit keine Informationen. Ein wichtiges Merkmal der Testqualit�t ist die sogenannte
Testabdeckung. Sie gibt an, wieviel des zu untersuchenden Quellcodes tats�chlich
durch die Tests ausgef�hrt wurde und insbesondere welche Teile des Quellcodes
nicht ausgef�hrt wurden.4


Das systematische Testen von Programmen ist ein wichtiger Bestandteil in der Entwicklung von Software, da es fast unm�glich ist fehlerfreie Software zu erzeugen. Durch das Testen sollen m�glichst viele Fehler gefunden werden. Allerdings existieren selbst bei relativ kleinen Programmen unendlich viele potentielle Testf�lle. Irgendwie muss eine Auswahl getroffen werden. Eine zuf�llige Auswahl der Testf�lle scheidet aus, da es sich dabei per Definition um eine unsystematische Auswahl handelt. Wichtige Fragen sind also: Welche Testf�lle soll ich w�hlen? Nach welchen Kriterien werden diese Testf�lle ausgew�hlt? 
Die verschiedenen Testarten - und damit auch die Auswahlkriterien f�r die Testf�lle - lassen sich grunds�tzlich in Black-Box- und Glass-Box-Verfahren (White-Box) unterteilen, wobei die Trennlinie nicht immer scharf ist.
Black-Box-Verfahren st�tzen sich nur auf die Programm-Spezifikation, der Programmcode an sich spielt keine Rolle bei der Auswahl der Testf�lle. Anders bei den Glass-Box-Verfahren, hier werden die Testf�lle durch die Programmstruktur - also den Code selbst - bestimmt.
In dieser Arbeit wird der Datenstrukturtest, bzw. sechs Auspr�gungen dieses Tests, etwas n�her betrachtet. Dieser Test - ein Glass-Box-Test - orientiert sich an den Zugriffen auf die Variablen, d.h. am Datenfluss und nicht am Kontrollfluss eines Programms.
   
   
