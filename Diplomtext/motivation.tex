In der heute sehr konkurenzbetonten Weltwirtschaft unterliegen die Gesch�ftsprozesse einer stetigen Ver�nderung: Unernehmen m�ssen st�ndig beobachten, wie sich die Marktbedingungen �ndern, und ihre Strategie ebtsprechend anpassen, um diese Ver�nderungen wiederzuspiegeln. Eine Schl�sselanforderung f�r eine moderne Enterprise IT ist also, dass Ver�nderungen in den IT-Systemen des Unternehmens schnell und effizient ber�cksichtigt werden k�nnen. Wichtige Motivation f�r die Einrichtung der SOA ist das Bestreben, die Agilit�t der Systemen der IT zu steigern. 
Unternehmen sehen sich heute mit sehr dynamischen M�arkten konfrontiert und sind oft
einem starken internationalen Wettbewerb ausgesetzt. Zu den Anforderungen solcher
M�arkte geh�oren oft flexible Gesch�aftsprozesse. Die notwendige Anpassungsf�ahigkeit der
Unternehmen und deren Gesch�aftsprozessen h�angt meistens direkt von der vorhandenen
IT-Infrastruktur ab. Um diesen Anforderungen gerecht zu werden muss die Softwarearchitektur
daf�ur sorgen, dass IT-Systemen zum einen an den Gesch�aftsprozessen ausgerichtet
sind, zum anderen sehr einfach bei Ver�anderungen angepasst werden k�onnen.
Viele setzen dabei auf SOA und Web Service Standards . 

Die Funktionen werden in Services gekapselt, die in Kompositionen zu vollautomatischen Gesch�ftsprozessen verkn�pft werden. F�r die Realisierung der Kompositionen steht mit WS-BPEL ein OASIS-Standard zu Verf�gung, der durch Unterst�tzung vieler wichtiger Softwarehersteller auf dem Weg zu einem Industriestandard ist. Damit �bernimmt BPEL  eine Schl�sselfunktion innerhalb der Konzeption einer serviceorientierten Architektur.

Die entscheidende Rolle der BPEL-Prozessen innerhalb eines Unternehmens erfordert hohe Qualit�tssicherung. Aufgrund der relativen Neuheit des ganzen Technologiebereichs gibt es in diesem Bereich noch wenige Erfahrungen. Viele Methoden, die sich in der konventionellen Softwareentwicklung bew�hrt haben,  sind noch nicht auf SOA �bertragen. W�hrend einige Konzepte und L�sungen f�r das systematische Testen von BPEL-Kompositionen bereits erarbeitet wurden, gibt es noch keine L�sungen zur Erfassung der Testabdeckung. 

Unter Testabdeckung versteht man Metriken f�r das Verh�ltnis zwischen zu testenden und tats�chlich getesteten Elementen des Pr�flings (in diesem Fall eine BPEL-Komposition). Diese Metriken werden  oft als Indikatoren f�r die Qualit�t und Fortschritt des Tests verwendet. 

In dieser Arbeit werden die Testabdeckungsmetriken f�r BPEL-Prozesse formal definiert. Dabei werden die bekannten Metriken (Anweisungs- und Zweigabdeckung) auf BPEL-Prozessen �bertragen und neue BPEL-spezifischen Metriken definiert, die die wichtigen Konzepten der Sprache abdecken. Es wird ein Konzept f�r die Messung der Testabdeckung erarbeitet und als Erweiterung des BPELUnit Frameworks implementiert.


 Allerdings bietet uns das neue Paradigma der Service Orientierten Architektur (SOA) zusammen mit den aktuellen, technologischen Fortschritten aus dem Umfeld der Web Services mittlerweile schon als sehr gut zu bezeichnende Unterst�tzung auf dem Weg zum flexiblen Echtzeitunternehmen. 