Wer benutz BPElUnit Was hat er davon

Wozu metriken f�r wenn


In einem Test-Prozess sind
Entwickler und Tester bestrebt m�glichst alle Teile des Quellcodes durch die Tests
mindestens einmal auszuf�hren. Ein JUnit-Test liefert jedoch als Ergebnis im wesentlichen
nur, ob er gelungen oder misslungen ist. �ber die Qualit�t eines Tests liefert
JUnit damit keine Informationen. Ein wichtiges Merkmal der Testqualit�t ist die sogenannte
Testabdeckung. Sie gibt an, wieviel des zu untersuchenden Quellcodes tats�chlich
durch die Tests ausgef�hrt wurde und insbesondere welche Teile des Quellcodes
nicht ausgef�hrt wurden.4


Das systematische Testen von Programmen ist ein wichtiger Bestandteil in der Entwicklung von Software, da es fast unm�glich ist fehlerfreie Software zu erzeugen. Durch das Testen sollen m�glichst viele Fehler gefunden werden. Allerdings existieren selbst bei relativ kleinen Programmen unendlich viele potentielle Testf�lle. Irgendwie muss eine Auswahl getroffen werden. Eine zuf�llige Auswahl der Testf�lle scheidet aus, da es sich dabei per Definition um eine unsystematische Auswahl handelt. Wichtige Fragen sind also: Welche Testf�lle soll ich w�hlen? Nach welchen Kriterien werden diese Testf�lle ausgew�hlt? 
Die verschiedenen Testarten - und damit auch die Auswahlkriterien f�r die Testf�lle - lassen sich grunds�tzlich in Black-Box- und Glass-Box-Verfahren (White-Box) unterteilen, wobei die Trennlinie nicht immer scharf ist.
Black-Box-Verfahren st�tzen sich nur auf die Programm-Spezifikation, der Programmcode an sich spielt keine Rolle bei der Auswahl der Testf�lle. Anders bei den Glass-Box-Verfahren, hier werden die Testf�lle durch die Programmstruktur - also den Code selbst - bestimmt.
In dieser Arbeit wird der Datenstrukturtest, bzw. sechs Auspr�gungen dieses Tests, etwas n�her betrachtet. Dieser Test - ein Glass-Box-Test - orientiert sich an den Zugriffen auf die Variablen, d.h. am Datenfluss und nicht am Kontrollfluss eines Programms.
   
   
Mit der Aufstieg der SOA und Web Service Stack of Standarts IT der Unternehmen haben sich in Eichtung der neu aufgabauten  und flexiblen Architekturen. Es sind Services implementiert, die in sogenannten Kompositionen eingeordnet zur Realisierung von vollautomatischen Gesch�gtsprozessen. Solche Prozesse operieren im Herz des Unternehmens. Deswegen ist die Technische Realisierung eine kritische Aufgabe.

The WS-BPEL standard ist Sprache zum Implementieren von Service Kompositionen. Weil BPEL bei OASIS standartisiert und von vielen Software Lieferanten unterst�tzt wird wird sie oft ausgew�hlt. Bevor BPEL-Prozess und die zugeh�rigen Web Services Eingesetzt werden k�nnen, das System muss getestet werden. um das Risiko zu verringern, die durch Fehler hervorgerufen wwird wegen der wichtigen Rolle im Unternehemen. Aber die Web Service und BPEL sind relativ neue technologiefelder und viele Methoden, die der herk�mliche Softwareentwicklung zur Verf�gung stehen sind noch nicht sind nicht auf SOA �bertragen oder es fehlen SOA basierten Erfahrungen. In dieser Arbeit wird die Testabdeckung des BPEL-Prozesses formal definiert. Testabdeckung ist Metrik wieviel der Software wurden ausgef�hrt bei einer Menge von Tests. Es wird oft benutzt um die Qualit�t der Tests zu antizipieren: Bereiche die nicht die nicht getestet wurden k�nnen eher Fehler aufweisen als die die getestet wurden        


N�chster logischer Schritt ist Messung der Testabdeckung auf BPEL zu �betragen. Es ist zu erwarten dass bei BPEL die Anzahl der Fehler sehr schnell steigt. 
\begin{itemize}
	\item BPEL befindet sich ganz oben auf dem Web Service Technologie Stack, der f�r die Unterst�tzung der Komplexen Gesch�ftsprozessen gedaxht ist.
	\item Es gibt noch keine Erfahrungen, und die Fallen sind unbekannt. 
	\item BPEL ist eng mit XML und XPath verbunden.
\end{itemize}
 
 
Testen von BPEL bedeutet �berpr�fen die Korrektheit des implementierten Gesch�ftsprozesses.
Das Hauptziel von BPELUnit Framework ist unit Test, die nat�rliche Benutzergruppe sind entwickler.