

   Visualisierung
   
   Beurteilung des Testqualit�ts ist in der Softwareentwicklung wichtig. Es wird noch Wichtiger, wenn ein kritisches System entwickelt wird. BPEL-Prozess geh�rt normalerweise zu dieser Kategorie. Als Ergebnis, die F�higkeit der Testabdeckungsmessung ist sehr n�tzlich. In dieser Arbeit pr�sentiert Testabdeckungsmetriken, die aus dem klassischen Softwareentwicklung auf BPEL adoptiert wurden und neue BPE-Spezifischen..
   
   Jede Metrik ist so definiert, dass die Tests f�r jeden BPEL-Prozess eine 100\% abdeckung erreichen k�nnen, was theoretisch ptimal ist. F�r die praktische Anwendung wurde BPELUnit Framework erweitert. Die Instrumentierung und Berechnung der Metriken erfolgt transparent beim Ablaufen der Tests.
   Die Erweiterung des BPELUnit Frameworks um die M�glichkeit die gemessene Testabdeckung zu visualisieren, ist der n�chste logische Schritt. Durch die Pr�sentation nicht abgedeckten Strukturen bekommt der Tester Hinweise, welche Tests noch fehlen und damit die M�glichkeit die Tests gezielt zu entwickeln. F�r das Erreichen der gew�nschten oder geforderter Testabdeckung ist diese Vorgehensweise viel effizienter und kann positive Auswirkungen auf die Entwicklungszeit und -Kosten haben.
   
   Anforderungsbasierte Testabdeckung.     