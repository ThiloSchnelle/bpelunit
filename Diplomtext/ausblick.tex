F�r BPEL-Prozesse, die meistens zu der unternehmenskritischen Software z�hlen, ist die Qualit�t der Tests sehr wichtig. Die Wahrscheinlichkeit, Fehler und M�ngel zu identifizieren, steigt mit zunehmendem Wert des Testabdeckungsma�es.

In dieser Arbeit wurden Codeabdeckungsmetriken f�r BPEL-Prozesse definiert:
\begin{itemize}
	\item Aktivit�tabdeckung: zeigt, wie viele Basisaktivit�ten bei den Tests ausgef�hrt wurden,
	\item Zweigabdeckung: zeigt, wie viele Zweige bei den Tests aktiviert wurde,
	\item Linkabdeckung: zeigt, wie viele Links bei den Tests ausgewertet wurden,
	\item Fault und Compensation Handler-Abdeckung: zeigt, wie viele Handler bei den Tests ausgel�st wurden.
\end{itemize}

Bei der Definition der Metriken wurde darauf geachtet, dass f�r jeden BPEL-Prozess 100\%-ge Abdeckung erreichbar ist. 

Implementiert wurde die Messung der Testabdeckung als Erweiterung des BPELUnit-Frameworks. Die Instrumentierung der BPEL-Dateien und Ermittlung der Testabdeckung erfolgt automatisch und transparent w�hrend eines Testlaufs. Es werden zwei Versionen der BPEL-Sprache unterst�tzt: 1.1 und 2.0. Die Erkennung erfolgt automatisch und erfordert keine zus�tzlichen Konfigurationen. Der Eclipse-Plugin und der Befehlszeilen-Client wurden entsprechend erweitert. Die beiden Clients bieten die M�glichkeit die Metriken festzulegen, diese zu konfigurieren und nach dem Testlauf die ermittelte Testabdeckung anzuzeigen.

Der n�chste logische Schritt ist die Visualisierung. Die grafische Darstellung der Testabdeckung kann den Testentwickler bei der gezielten Testerstellung besser unterst�tzen.
F�r das Erreichen der gew�nschten oder geforderter Testabdeckung ist diese Vorgehensweise viel effizienter und kann positive Auswirkungen auf die Entwicklungszeit und -Kosten haben.

Interessant w�re eine Erweiterung des BPELUnit-Frameworks, die die Ausf�hrung mehrerer Testsuites erm�glichen w�rde. Damit w�re es auch m�glich die Testabdeckung �ber mehrere \textit{Suites} zu ermitteln. 

Ein weiterer sehr interessanter Aspekt ist die Unterst�tzung der Anforderungsabdeckung f�r BPEL-Prozesse und die Definition der anforderungsbasierten (\textit{black box}-) Tesdtabdeckungsmetriken.

Die implementierte Erweiterung des BPELUnit-Frameworks bietet eine einfache M�glichkeit, die Komplexit�t des Prozesses zu berechnen. Eine entsprechende Anpassung der Definition von Komplexit�t f�r BPEL wurde in \cite{Cardoso2006} vorgeschlagen.


