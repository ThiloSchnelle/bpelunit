Die vorliegende Arbeit gliedert sich neben der Einleitung in 8 Kapitel auf.

In Kapitel 2 werden die theoretischen Grundlagen aufgef�hrt, auf denen diese Masterarbeit aufbaut.
Unter anderem wird der Serviceorientierten Architekturen definiert, Web Services mit den zugeh�rigen Standards kurz vorgestellt und die f�r den weiteren Verlauf dieser Arbeit relevante Kompositionssprache BPEL (Business Process Execution Language) mit den wichtigen Konzepten ausf�hrlich erl�utert. Au�erdem werden die Grundlagen des Testens von BPEL-Kompositionen und das Test-Framework BPELUnit vorgestellt. Anschlie�end wird der Begriff Testabdeckung definiert und die Testabdeckungsmetriken Anweisungs- und Zweigabdeckung vorgestellt.

In Kapitel 3 werden die Metriken f�r die Messung der Testabdeckung in BPEL definiert.
Dabei werden die Anweisungs- und Zweigabdeckung an den BPEL-Kontext angepasst
und neue BPEL-spezifische Metriken, Link- und Handlerabdeckung, eingef�hrt.

In Kapitel 4 wird ein geeignetes Verfahren f�r die Ermittlung der Testabdeckung in BPEL ausgew�hlt. Zuvor werden bekannte Ans�tze vorgestellt, analysiert und bewertet. 

In Kapitel 5 wird die Umsetzung der Ermittlung von Testabdeckung f�r die einzelnen Metriken vorgestellt. 
Einige Aspekte des Designs und der Implementierung sowie die Integration der L�sung in das BPELUnit-Framework werden in Kapitel 6 vorgestellt. 


 In Kapitel 7 werden die Testabdeckungsmetriken an einem Beispiel erl�utert. Anschlie�end im Kapitel 8  wird die praktische Einsatzf�higkeit des Verfahrens und der Implementierung in einer Fallstudie �berpr�ft.

Abschlie�end in Kapitel 9 werden die wesentlichen Ergebnisse der Arbeit zusammengefasst und ein Ausblick �ber m�gliche Erweiterungen gegeben.
