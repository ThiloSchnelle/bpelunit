Durch die Berechnung der Testabdeckung will der Tester erfahren, wieviel von dem Pr�fling beim Testen ausgef�hrt wurde. F�r die Definition des Pr�flings, in diesem Fall BPEL-Prozesses, wird die BPEL Formalisierung von Ouyang in ??? verwendet. Nach der Pr�sentation der wichtigsten Punkten dieser Formalisierung folgen die Definition der  Aktivit�t- und Zweigabdeckung. Dar�ber hinaus werden zwei spezial f�r BPEL definierten Metriken vorgestellt: Link- und Handlerabdeckung.  Anschlie�end wird das Konzept f�r die Umsetzung und Integration in das BPELUnit- Framework vorgestellt.
\section{Formale Definition von BPEL}
Die hier vorgestellte formale Beschreibung von WS-BPEL Process Model ist nicht
vollst�ndig und fasst nicht den gesamten Umfang von WS-BPEL um. Es werden nur
f�r diese Arbeit relevante Teile des Modells beschrieben. Als Grundlage dient
eine Definition aus  \cite{Ouyang2005}, die abgesehen von der Anpassung an den
Standard WS-BPEL 2.0 und einigen Erweiterungen, unver�ndert �bernommen wurde.  

\begin{Def}[WS-BPEL Process Model~\cite{Ouyang2005}] A WS-BPEL Process
Model is a tuple $\mathcal{W=(A,\ E,\ C,\ L,\ HR,\ }type_{\mathcal{A}},\ type_{\mathcal{E}},\
instance,\ name,\ <_{seq},\ <_{if},\\ serialscp,\ process,\ trigger,\ scp_c,\
trigger_c,\ scp_t,\ trigger_{tf},\ \mathcal{LR},\ joincon,\\ 
transitionCondition,\ supjoinf,\ trigger_{jf})$ where: 
\end{Def}

\begin{itemize}
  \item $\mathcal{A}$ is a set of activities,
  \item $\mathcal{E}$ is a set of events,
  \item $\mathcal{C}$ is a set of conditions,
  \item $\mathcal{L}$ is a set of control links,
  \item let $\mathcal{B=E\cup C} \cup\{\bot\} $ be a set of labels where $\bot$
  denotes the empty label, then $\mathcal{HR\subseteq A\times B\times A}$ is a labeled tree which defines the relation between an activity and its direct sub-activities,
	\item $\forall a\in \mathcal{A},\ let\ \mathcal{HR}_{p}=_{\pi 1,3}\mathcal{HR}$(the projection of $\mathcal{HR}$ on two activity sets), $children(a)=\{a'\in \mathcal{A}|\mathcal{HR}_p(a,a')\}$ is the set of immediate descendants of $a$,
	\item $type_\mathcal{A}:\mathcal{A\rightarrow T_A}$ is a function that assigns types to activities taken from the set of activity types 
	
	\begin{itemize}
		\item $\mathcal{T_A=T_B\cup T_S}$
		\item $\mathcal{T_B}=\{receive,\ reply,\ wait,\ assign,\ validate,\ empty,\
		throw,\ rethrow,\\ \ compensate,\ compensateScope,\ exit\}$
 		\item $\mathcal{T_S}=\{sequence,\ flow,\ pick,\ if,\ while,\ repeatUntil,\
		forEach,\ scope\}$
    \end{itemize}
    
	\item $\forall t\in \mathcal{T_A},\ \mathcal{A}_t=\{a\in \mathcal{A}|type_\mathcal{A}(a)=t\}$ is a set of all activities of type $t$,
 	\item $type_\mathcal{E}:\mathcal{E\rightarrow T_E}$ is a function that assigns
 	types to events taken from the set of event types $\mathcal{T_E}=\{message,\ alarm,\ fault,\ compensation,\ termination\}.$
 	\item $\forall t\in \mathcal{T_E},\ \mathcal{E}_t=\{e\in \mathcal{E}|type_\mathcal{E}(e)=t\}$ is a set of all events of type t,
%\item $instance:\mathcal{A}_{receive}\cup \mathcal{A}_{pick}\rightarrow \mathcal{B}$ is a %function which sddigns a boolean value to the createInstance attribete of a receive or a %pick activity.
\item $let\ \mathcal{A}^{structured}=\mathcal{A}_{sequence}\cup \mathcal{A}_{flow}\cup \mathcal{A}_{if}\cup \mathcal{A}_{while}\cup \mathcal{A}_{repeatUntil}\cup \mathcal{A}_{forEach}\cup \mathcal{A}_{pick}\cup \mathcal{A}_{scope}$ be a set of structured activities, $\forall_{s\in \mathcal{A}^{structured}}(children(s)\neq \emptyset)$, i.e., they are the internal nodes of the $\mathcal{HR}$ tree,
\item let $\mathcal{A}^{basic}=\mathcal{A}_{invoke}\cup \mathcal{A}_{receive}\cup \mathcal{A}_{reply}\cup \mathcal{A}_{wait}\cup \mathcal{A}_{assign}\cup \mathcal{A}_{validate}\cup \mathcal{A}_{empty}\cup \mathcal{A}_{validate}\cup \mathcal{A}_{throw}\cup \mathcal{A}_{rethrow}\cup \mathcal{A}_{compensate}\cup \mathcal{A}_{compensateScope}$ be a set of basic activities, $\forall_{s\in \mathcal{A}^{basic}}(children(s)= \emptyset)$, i.e., they are the leaves of the $\mathcal{HR}$ tree,
\item given $\mathcal{A'}=\mathcal{A}_{sequence}\cup \mathcal{A}_{flow},\
\mathcal{HR} \cap (\mathcal{A}'\times B\times \mathcal{A})=\mathcal{HR} \cap
(\mathcal{A'}\times \{\bot\}\times \mathcal{A})$, which represents the automatic passing of the control-flow from
an activity to its sub-activities,
\item $\forall s\in \mathcal{A}_{sequence},\exists$ an order $<^s_{seq}$ which is strict total order over $children(s)$,
\item $\mathcal{HR}\cap (\mathcal{A}_{pick}\times \mathcal{B}\times \mathcal{A})=\mathcal{HR}\cap (\mathcal{A}_{pick}\times \mathcal{E}^{normal}\times \mathcal{A})$, where $\mathcal{E}^{normal}=\mathcal{E}_{message}\cup \mathcal{E}_{alarm}$ provides a set of normal events,
\item given $\mathcal{A'}=\mathcal{A}_{if}\cup \mathcal{A}_{while}\cup
\mathcal{A}_{forEach},\ \mathcal{HR}\cap (\mathcal{A'}\times \mathcal{B}\times
\mathcal{A})=\mathcal{HR}\cap (\mathcal{A'}\times \mathcal{C}\times
\mathcal{A})$, so that if, while and for-each activities must evaluate a condition,
\item given $\mathcal{A'}=\mathcal{A}_{repeatUntil},\ \mathcal{HR}\cap (\mathcal{A'}\times \mathcal{B}\times \mathcal{A})=\mathcal{HR}\cap (\mathcal{A'}\times \{\bot\}\times \mathcal{A}))$,
\item $\forall s\in \mathcal{A}_{if},\ \exists$ an order $<^{s}_{if}$ which is a strict total order over $children(s)$,
\item $\forall s\in \mathcal{A}_{if}$, let $last(s)\in children(s)$ be the
sub-activity in the last branch evaluated in s such that $\neg \exists_{a\in
children(s)}(last(s)<^{s}_{if}a)$, let $c\in \mathcal{C},\\
\mathcal{HR}(s,c,last(s))\Rightarrow \forall_{assign(c)\in
Assign(\mathcal{C})}eval(c,assign(c))=true$. Note that $last(s)$ represents the
else branch in if activity, which ensures that (a) an else branch always exists
and (b) at least one of the branches is taken in the activity,
\item $\forall s\in \mathcal{A}_{while},\ \left| \mathcal{HR}\cap (\{s\}\times \mathcal{C}\times \mathcal{A})\right|=1$, i.e., each while activity has exactly one sub-activity,
\item $\mathcal{HR}\cap (\mathcal{A}_{scope}\times \mathcal{B}\times \mathcal{A})=\mathcal{HR}\cap (\mathcal{A}_{scope} \times (\mathcal{E}\times\{\bot \})\times \mathcal{A})$, where: $\forall s \in \mathcal{A}_{scope}$,
\begin{itemize}
	\item $\left|\mathcal{HR}\cap (\{s\}\times \{\bot\}\times \mathcal{A})\right|=1$, i.e., each scope has one primary (or main) activity,
	\item $\left|\mathcal{HR}\cap (\{s\}\times \{\mathcal{E}_{fault}\}\times \mathcal{A})\right|\geq 1$, i.e., each scope provides at least one fault handler,
		\item $\left|\mathcal{HR}\cap (\{s\}\times \{\mathcal{E}_{compensation}\}\times \mathcal{A})\right|\leq 1$, i.e., each scope provides at most one compensation handler,
\end{itemize}
\item $process\in \mathcal{A}_{scope}$ is the root of the $\mathcal{HR}$ tree,
\item $trigger_{tf}:\mathcal{A}_{throw}\cup \mathcal{A}_{rethrow}\rightarrow \mathcal{E}_{fault}$ is a function which maps each throw activity to a (process-defined) fault event triggered by that activity,
\item $scp_c:\mathcal{E}_{compensation}\rightarrow \mathcal{A}_{scope}\backslash \{process\}$ is an injective function mapping a compensation event to a (non-process) scope such that the occurrence of that event invokes the compensation of that scope,
\item $trigger_c:\mathcal{A}_{compensate}\rightarrow \mathcal{E}_{compensation}$ is an injective function which maps each compensate activity
to a compensation event triggered by that activity,
\item $\mathcal{LR}\subseteq \mathcal{A}\times \mathcal{L}\times \mathcal{A}$ is a labeled acyclic graph which defines the relation between the source
activity of a control link and the target activity of the link,
\item let $\mathcal{A}^{source}=\{a\in \mathcal{A}|\exists_{l\in \mathcal{L}}((a,l)\in\pi_{1,2}\mathcal{LR})\}$ be a set of source activities of all control links, and $\mathcal{A}^{target}=\{a\in \mathcal{A}|\exists_{l\in \mathcal{L}}((l,a)\in \pi_{2,3}\mathcal{LR})\}$ be a set of target activities of all control links, then $\forall a\in \mathcal{A}^{source},\mathcal{L}_{out}(a)=\\
\{l\in \mathcal{L}|\exists_{a'\in \mathcal{A}}\mathcal{LR}(a,l,a')\}$ is a set of all outgoing control links from a,
and $\forall a\in \mathcal{A}^{target}, \mathcal{L}_{in}(a)=\{l\in \mathcal{L}|\exists_{a'\in \mathcal{A}}\mathcal{LR}(a',l,a)\}$ is a set of all incoming control links to $a$,
\item let $a\in \mathcal{A}^{target},\ joincon(a)$, which expresses the join condition of incoming control links at $a$, is a
boolean function over $\mathcal{L}_{in}(a)$(i.e. $Var(joincon(a))=\mathcal{L}_{in}(a))$,
\item let $l\in_{ \pi 2}\mathcal{LR},\ transitionCondition(l)$, which expresses the transition condition of links, is a boolean function,
\item Dead-path-elimination (DPE). 
Wird eine Aktivit�t aufgrund einer zu $false$ ausgewerteten $joinCondition$ oder eines nicht abgearbeiteten Zweiges einer $if-$ oder $pick-$Aktivit�t nicht ausgef�hrt, so wird f�r alle ausgehenden Links die $transitionCondition$ auf $false$ gesetzt. Dieses Verhalten wird als Dead-Path-Elimination bezeichnet. 
\end{itemize}

Betrachtet man die Kontrollstruktur eines WS-BPEL Prozesses, so kann die $\mathcal{HR}$-Relation als �bergabe der Kontrolle von einer Aktivit�t an ihre Subaktivit�ten interpretiert werden. Damit entspricht ein Element dieser Relation einem Zweig des Kontrollflussgraphen. Allerdings deckt diese Relation nicht alle Zweige des Graphen ab, es fehlen n�mlich die Zweige, die die Kontrolle innerhalb der Schleifen von der inneren Aktivit�t an die Schleife zur�ckgeben. Die folgende Relation $\mathcal{HBR}$ beschreibt genau solche Beziehungen:\begin{itemize}
	\item $\mathcal{HBR}\subseteq \mathcal{A}\times(\mathcal{C}\cup
	\{\bot\})\times \mathcal{A}$ describes the relation of an activity to its
	parent activity,
	\item $\forall a,a'\in \mathcal{A},\ \forall(a,a')\in_{\pi
	1,3}\mathcal{HBR}\Rightarrow a\in children(a')$ 
\end{itemize}

Consequently, for all loops holds:
\begin{itemize}
	\item given $\mathcal{A'}=\mathcal{A}_{while}\cup \mathcal{A}_{forEach},\
	\mathcal{HBR}\cap (\mathcal{A}\times \mathcal{B}\times
	\mathcal{A'})=\mathcal{HBR}\cap (\mathcal{A}\times \{\bot \}\times 
	\mathcal{A'}),$   
	\item given $\mathcal{A'}=\mathcal{A}_{repeatUntil},\ \mathcal{HBR}\cap
	(\mathcal{A}\times \mathcal{B}\times \mathcal{A'})=\mathcal{HBR}\cap
	(\mathcal{A}\times \mathcal{C}\times \mathcal{A'}).$
\end{itemize}


Die Relationen $\mathcal{HR}$ und $\mathcal{HBR}$ beschreiben zusammen alle Kanten des Kontrollflussgraphen des zugeh�rigen WS-BPEL Prozesses.
