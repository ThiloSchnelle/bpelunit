In diesem Kapitel wurden die allgemeinen Grundlagen behandelt, die f�r die Definition der Metriken und Ermittlung der Testabdeckung in BPEL-Kompositionen notwendig sind. Nach der Vorstellung der allgemeinen Definition der Serviceorientierten Architekturen wurden Web Services mit den zugeh�rigen Standards erl�utert. Etwas ausf�hrlicher wurde WSDL-Standard vorgestellt, der f�r die Beschreibung der Web Service-Schnittstelle eingesetzt und in dieser Arbeit verwendet wird. Anschlie�end wurde die BPEL-Sprache, die im Fokus dieser Arbeit steht, ausf�hrlich behandelt. Neben der BPEL-Aktivit�ten, die f�r die Realisierung der Gesch�ftslogik
verwendet werden, wurden das Link-Konzept, die Fehlerbehandlung und Kompensation vorgestellt. Die detaillierte Beschreibung der Konzepten der Konzepten der Sprache dient dem Vers�ndnis der formalen Definition der Metriken (Kapitel \ref{chap:testabdceckungInBPEL}) und der Umsetzung der Abdeckungsmessung (Kapitel \ref{chap:umsetzung}).

Mit der Testabdeckung in BPEL-Kompositionen behandelt diese Arbeit einen weiter gehenden Aspekt des Testens von BPEL-Prozessen. Dementsprechend wurden die Grundlagen des Testens von BPEL-Kompositionen ebenfalls in diesem Abschnitt behandelt. Insbesondere wurde dabei auf das BPELUnit Framework eingegangen, das das Testen von BPEL-Kompositionen unterst�tzt und in dieser Arbeit durch die Testabdeckungsmetriken erweitert wird.

Zum Schluss wurde der Begriff Abdeckung erl�utert und zwei wichtigen Abdeckungsmetriken Anweisungs- und Zweigabdeckung vorgestellt.

