  Unternehmen sehen sich heute mit sehr dynamischen M�rkten konfrontiert und sind oft einem starken internationalen Wettbewerb ausgesetzt. Zu den Anforderungen solcher M�rkte geh�ren oft flexible Gesch�ftsprozesse. Die notwendige Anpassungsf�higkeit der Unternehmen und deren Gesch�ftsprozessen h�ngt meistens direkt von der vorhandenen IT-Infrastruktur ab. Um diesen Anforderungen gerecht zu werden muss die Softwarearchitektur zum einen daf�r sorgen, dass IT-Systemen an den Gesch�ftsprozessen ausgerichtet, zum anderen sehr einfach bei Ver�nderungen angepasst werden k�nnen. SOA verspricht diese F�higkeit zur flexiblen und effektiven Unterst�tzung sich immer schneller �ndernder Gesch�ftsprozesse. 
  
  SOA(Service Oriented Architecture) ist ein Ansatz zum Entwurf verteilter Systeme. Die Kernidee der SOA, die Unternehmensfunktionalit�t
als Menge voneinander unabh�ngiger Dienste zur Verf�gung
zu stellen, bildet die Basis f�r Integration und Dynamik \cite{THOMAS2005}. Eine m�gliche Definition f�r SOA:
\begin{Def}Eine serviceorientierte Architektur ist ein
Konzept f�r eine Systemarchitektur, in
welchem Funktionen in Form von wieder
verwendbaren, voneinander unabh�ngigen
und lose gekoppelten Services implementiert
werden. Services k�nnen unabh�ngig von
zugrunde liegenden Implementierungen �ber
Schnittstellen aufgerufen werden, deren
Spezifikationen �ffentlich sind. Serviceinteraktion
findet �ber eine daf�r vorgesehene
Kommunikationsinfrastruktur statt.
Quelle Wikipedia 23.01.2006
\end{Def} 
  Die Funktionen einer Anwendung sind als Service organisiert, die 
 beliebig verteilt sein k�nnen und lassen sich dynamisch zu Gesch�ftsprozessen verbinden. Die zugrunde liegenden technischen Plattformen der einzelnen Services spielen dabei keine Rolle. Zu betonen ist, dass SOA eine Systemarchitektur und keine Technologie beschreibt. 


  Die wichtigen Eigenschaften sind:
  
  
\begin{itemize}
	\item Service
	\item wiederverwendbar
	\item unabh�ngig
	\item lose gekoppelt
	\item Schnittstellen unabh�ngig von der Impementierung
\end{itemize}


Prim�rziel ist, die historisch gewachsene, heterogene Systemlandschaft effizient an �nderungen im Gesch�ftsprozess anpassen zu k�nnen. Im Einzelnen soll dadurch Software erstellt werden, die

    * einfach an neue Bed�rfnisse angepasst werden kann (Flexibilit�t)
    * wiederverwendbar ist
    * verteilt installiert werden kann
    * an Gesch�ftsprozesse angepasst ist.

Sekund�rziele sind:

    * Kostenvorteile durch schnelle Optimierung
    * h�here Produktivit�t der Softwareentwickler durch Wiederverwendung von Services
    * schnelle Reaktion auf Herausforderungen m�glich
    * mittelfristige Einsparungen
    * Reduzierung der Komplexit�t durch Aufbrechen monolithischer IT-Systeme
    * schrittweise Restrukturierung komplexer Anwendungssysteme.

