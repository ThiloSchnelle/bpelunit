Die Unternehmen sind heutzutage an agilen IT-Systemen interessiert, die die Gesch�ftsprozesse unterst�tzen und flexibel auf Ver�nderungen reagieren k�nnen.  
Serviceorientierte Architekturen (SOA) verspricht diese F�higkeit zur effektiven und flexiblen Unterst�tzung sich immer schneller �ndernder Gesch�ftsprozesse. 
  
  SOA ist ein Ansatz zum Entwurf verteilter Systeme. \textit{Die Kernidee der SOA, die Unternehmensfunktionalit�t
als Menge voneinander unabh�ngiger Dienste zur Verf�gung
zu stellen, bildet die Basis f�r Integration und Dynamik} \cite{THOMAS2005}. Eine m�gliche Definition f�r SOA:
\begin{quotation}
\begin{Def}\textit{Eine serviceorientierte Architektur ist ein
Konzept f�r eine Systemarchitektur, in
welchem Funktionen in Form von wiederverwendbaren, voneinander unabh�ngigen
und lose gekoppelten Services implementiert
werden. Services k�nnen unabh�ngig von
zugrunde liegenden Implementierungen �ber
Schnittstellen aufgerufen werden, deren
Spezifikationen �ffentlich sind. Serviceinteraktion
findet �ber eine daf�r vorgesehene
Kommunikationsinfrastruktur statt.}
\cite{WikiSOA}
\end{Def}
\end{quotation}

  Die Funktionen einer Anwendung sind als Service organisiert, die 
 beliebig verteilt sein k�nnen und sich dynamisch zu Gesch�ftsprozessen verbinden lassen. Die zugrunde liegenden technischen Plattformen der einzelnen Services spielen dabei keine Rolle. Zu betonen ist, dass SOA eine Systemarchitektur und keine Technologie beschreibt. 

 