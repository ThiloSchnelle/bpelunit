  SOA steht f�r Service Oriented Architecture, zu Deutsch auch Service-orientierte Architektur, ein Ansatz zum Entwurf verteilter Systeme im Softwaredesign. Es handelt sich um eine Weiterentwicklung des Web-Services-Paradigma, wobei SOA eine Anzahl von Services in einem Netzwerk beschreibt, das mit anderen kommunizieren. Ein typisches Merkmal f�r SOA ist die standardbasierte Integrationen via XML und/oder Web Services bei der Konsolidierung von heterogenen Systemen.
   Darin lassen sich Softwareservices erstellen, verwalten und kombinieren. Das gro�e Ziel ist eine an Gesch�ftsprozessen ausgerichtete IT-Infrastruktur, die schnell auf ver�nderte Anforderungen reagiert. Weil Services mehrfach verwendet werden k�nnen, verspricht SOA zudem Kostenvorteile.
   es sich dabei um ein Konzept handelt, mit dem Software so modular und standardisiert entwickelt wird, dass die Funktionen als f�r jede Anwender verst�ndliche Services etwa �ber ein Portal abrufbar sind. Der Clou: Diese Funktionen lassen sich je nach ver�ndertem Gesch�ftsbed�rfnis neu konfigurieren
   
   
   <Microsoft>SOA beschreibt eine Software-Infrastruktur, in der die wesentlichen Funktionen einer Anwendung bzw. Softwaremodule als Service organisiert sind. Services k�nnen beliebig verteilt sein und lassen sich dynamisch zu Gesch�ftsprozessen verbinden. SOA legt hierbei die Schnittstellen fest, �ber die andere Systeme via Netzwerk diese Dienste nutzen k�nnen.
Services tauschen dadurch unabh�ngig von den zugrunde liegenden technischen Plattformen Daten aus. Zwingende Abh�ngigkeiten der monolithischen Architekturen und zwischen bestimmten Client-/Server-Architekturen sind damit aufgel�st.