Die Beschreibung enth�lt Informationen dar�ber , woraus der Web Service besteht, welche Daten ausgetauscht werden und wie der Web Service aufgebaut ist.
WSDL dient dem Webservice dazu, standardisiertbekannzugeben, welche Funktionen er besitzt, wie die Schnittstellen (Parameter) aussehen und �ber welche Protokolle Konsumenten ihn ansprechen k�nnen.
 Seit M�rz 2001 ist die Version 1.1 aktuell. Auch hier wurde eine Arbeitsgruppe im W3C zur
Weiterentwicklung gegr�ndet, WSDL 1.2 und 2.0 sind in Arbeit.
Versionen 


Damit das Format hinsichtlich zuk�nftiger Nachrichtenformate und Transportprotokolle erweiterbar
ist, gliedert es sich in zwei Teile. Eine abstrakte und eine konkrete Definition.
Die abstrakte Definition beschreibt die Datentypen, die Operationen und Nachrichten. Diese
werden zu so genannten portTypes zusammen gefasst.
Die konkrete Definition bindet die portTypes an ein Transportprotokoll und fasst diese
ports zu services zusammen.
Dokument Struktur
Abstrakte Definition Konkrete Definitionen
Operationen undMessageswerden erstabstrakt beschrieben und dannan das jeweilige Netzwerk-Protokoll undMessage-Format gebunden (SOAP, HTTP, MIME)
Bild

definitions ist das Wurzelelement des WSDL-Dokuments. Es enth�lt den Namen des Services, Namespaces f�r den Service selbst und f�r die verwendete Standards.

types enth�lt Definitionen f�r eigene Datentypen. Standardm��ig werden die Datentypen aus der XML-Schema-Spezifikation verwendet.
Datentypen dienen derInteroperabilit�tvon Anwendungen und Diensten, um eine gemeinsame Austauschbasis f�r Daten zu gew�hrleisten

Das Konzept der XML-Schemas ist
plattformneutralundsprachunabh�ngig
Die Flexibilit�tder XML-Schemas erlaubt es in WSDL die Besonderheiten bestehender Programmiersprachen oder Datenaustauschstandards abzubilden


message  fasst die definierten Typen zu abstrakten Nachrichten zusammen
definiert nur unidirektionale Nachrichten

portType Der portType beschreibt die Schnittstelle des Services. In ihm werden eine oder mehrere
Operationen zusammen gefasst, �hnlich wie ein Java Interface. Diese ordnen der Operation
die Ein- und Ausgabeparameter zu. �ber diese Zuordnung wird das sogenannte Message
Exchange Pattern der Schnittstelle festgelegt.

binding die vom Web Service verwendeten Kommunikationsprotokolle und Nachrichten formate. definiert konkrete Nachrichtenformate und Protokoll-Einzelheiten
Die meist benutzten Protokolle sind SOAP, HTTP GET/POST. 
Eine Bindung muss exakt ein Protokoll spezifizieren. 
Jede Bindung baut auf genau eine Port Type auf.


service definiert die Endpunkte eines Dienstes. Der Endpunkt selbst steht mit einer URI im untergeordneten port-Element. 
Das service-Element stellt aus einem oder mehreren ports einen Dienst zusammen.
Ein port ordnet dem Binding einen bestimmten Endpunkt zu. Wie dieser definiert wird,
h�ngt von dem verwendeten Transportprotokoll ab. Bei SOAP �ber HTTP ist dies beispielsweise
die URL, �ber die der Service Provider erreichbar ist.






  
  
  WSDL ist jedoch nicht allein an SOAP gebunden. Es gibt eine Vielzahl von vordefinierten Bindungen, zu denen auch SOAP 1.1 z�hlt. WSDL h�lt sich an eine abstrakte Beschreibung von Web Services. WSDL verfolgt somit unterschiedliche Ziele:

		
 	Jeder Web-Service l�a�t sich als Objekt verstehen, das verschiedene Methoden
zur Verf�ugung stellt, auf die zugegriffen werden kann. Um diese Zugriffe zu
erm�oglichen und zu automatisieren, ist eine standardisierte Beschreibung der
Schnittstelle zum Web-Service notwendig. Hierf�ur wird die Web-Service Description
Language verwendet.
WSDL ist eine XML-basierte Sprache zur Beschreibung von Web-Services.
Sie beschreibt die Methodenschnittstelle desWeb-Services, nicht aber denWeb-
Service selbst. Neben der Methodenspezifikation beinhaltet sie auch technische
Daten, wie die eigentliche Lage des Dienstes und an welche Transportprotokolle
er gebunden ist.
DieWeiterentwicklung von WSDL wird durch das W3C kooridiniert. WSDL
liegt aktuell in der Version 2.0 vor und gliedert sich in drei Teile.
Die Core Language ([WSDL Pt. 1]) mit allen wesentlichen Strukturierungselementen,
die Message Exchange Patterns ([WSDL Pt. 1]) mit der Definition der
m�oglichen Kommunikationsmuster und
die Bindings, welche den Zugriff auf Operationen und den Transport der
gesendeten Nachrichten beschreiben.

Die Elemente in einem WSDL-Dokument lassen sich in abstrakte Definitionen
und konkrete technische Beschreibungen zu diesen Definitionen unterteilen. Die
abstrakten Elemente dienen dabei der detaillierten Beschreibung der Schnittstelle
einer Softwarekomponente. Die technischen Elemente dagegen beschreiben,
wo Endpunkte der Komponente im Netzwerk lokalisiert und �uber welche
Protokolle sie angesprochen werden k�onnen.
Die Unterscheidung in abstrakte und konkrete Definitionen bedeutet dabei
aus Sicht der Middleware eine Trennung zwischen Schnittstellenbeschreibung
und Serialisierungs- und Transportdetails. Sie erh�oht die Erweiterbarkeit, Wiederverwendbarkeit
und Portabilit�at von Web-Services.
Die konkreten Definitionen referenzieren
dabei stets abstrakte Definitionen. Das bedeutet, da� zu einer abstrakten
Definition mehrere konkrete Implementierungen existieren k�onnen.
Abbildung 6 zeigt die allgemeine Struktur eines WSDL-Dokuments und
die Verkn�upfungen der einzelnen Teilkomponenten. Jedes WSDL-Dokument
wird von einem definitions-Element eingeleitet. Das Element definiert den
Namensraum des WSDL-Dokuments und �offnet alle weiteren ben�otigten Namensr
�aume.
Abstrakte Definition
Die abstrakte Definition beinhaltet die sprach- und plattformunabh�angige Beschreibung
der Schnittstelle des Web-Service. Zu den abstrakten Definitionen
z�ahlen die types- und interface-Elemente.
Innerhalb des types-Elements werden die verwendeten Typen definiert. 
Der konkrete Definitionsteil beinhaltet, an welcher Stelle ein Web-Service �uberhaupt
zu finden und durch welche Transportmechanismen er abrufbar ist. Die
Definition umfa�t die binding- und service-Elemente.
Das binding-Element ist ein Container f�ur Elemente zur Beschreibung der konkreten
Serialisierungs- und Transportdetails f�ur ein deklariertes Interface.




