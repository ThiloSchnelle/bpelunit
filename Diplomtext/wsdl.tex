WSDL ist eine XML-basierte Sprache zur Beschreibung von Web-Services.
Sie beschreibt die Schnittstelle des Web-Services, nicht aber den Web-
Service selbst. Neben der Methodenspezifikation beinhaltet sie auch technische
Daten, wie die eigentliche Lage des Dienstes und an welche Transportprotokolle
er gebunden ist.
Seit M�rz 2001 ist die Version 1.1 aktuell. Auch hier wurde eine Arbeitsgruppe im W3C zur
Weiterentwicklung gegr�ndet, WSDL 1.2 und 2.0 sind in Arbeit.


Die Struktur eines WSDL-Dokuments ist in ... abgebildet.
Bild..
Die Elemente in einem WSDL-Dokument lassen sich in abstrakte Definitionen
und konkrete technische Beschreibungen zu diesen Definitionen unterteilen. 
Die sprach- und plattformunabh�ngige Beschreibung
der Schnittstelle (Datentypen, Operationen und Nachrichten) finden in den abstrakten Elemente statt: types-, message und portType-Elemente.
Die konkreten technischen Details (wo die Services sich befinden und wie sie aufgerufen werden k�nnen) werden durch die Elemente binding und service definiert.  Die konkreten Definitionen referenzieren
dabei stets abstrakte Definitionen. Das bedeutet, da� zu einer abstrakten
Definition mehrere konkrete Implementierungen  existieren k�nnen. Damit besteht die M�glichkeit ein Service an mehreren location verf�gbar zu machen und die Verwendung verschiedenen Kommunikations- und Transportprotokollen zu erm�glichen.

Die Elemente

definitions ist das Wurzelelement des WSDL-Dokuments. Es enth�lt den Namen des Services, Namespaces f�r den Service selbst und f�r die verwendeten Standards.

types enth�lt Definitionen f�r eigene Datentypen. Standardm��ig werden die Datentypen aus der XML-Schema-Spezifikation verwendet. Das Konzept der XML-Schemas ist
plattformneutral und sprachunabh�ngig
Die Flexibilit�tder XML-Schemas erlaubt es in WSDL die Besonderheiten bestehender Programmiersprachen oder Datenaustauschstandards abzubilden


message  fasst die definierten Typen zu abstrakten Nachrichten zusammen
definiert nur unidirektionale Nachrichten

portType Der portType beschreibt die Schnittstelle des Services, die eine oder mehrere
Operationen zusammenfasst. Diese ordnen der Operation
die Ein- und Ausgabeparameter zu. �ber diese Zuordnung wird das Nachrichtenaustauschmuster der Schnittstelle festgelegt.

binding definiert die vom Web Service verwendeten Kommunikationsprotokolle und Nachrichten formate. Die meist benutzten Protokolle sind SOAP, HTTP GET/POST. 

service Das service-Element stellt aus einem oder mehreren ports einen Dienst zusammen.
Ein port ordnet dem Binding einen bestimmten Endpunkt zu. Die genaue Definition h�ngt vom verwendeten Transportprotokoll ab.




Die vorliegende Arbeit setzt auf WSDL auf, daher hier eine kurze Einf�uhrung. Derzeit ist
die Version 1.1 g�ultiger Standard, es wird aber bereits an der Version 2.0 gearbeitet. Die
wesentlichen Konzepte einer WSDL-Schnittstellenbeschreibung in der Version 1.1 sind in
Abbildung 2.3 dargelegt. Diese lassen sich prinzipiell in drei Kategorien einteilen: Schnittstelle,
Bindung und Services.



Wie das W3C bekannt gegeben hat, ist die Version 2 der Web Services Description Language (WSDL) als vorgeschlagene Empfehlung (Proposed Recommendation) in drei Teilen ver�ffentlicht worden: Teil 0: Primer, Teil 1: Core Language und Teil 2: Adjuncts. Anmerkungen und Kommentare k�nnen bis 20. Juni abgegeben werden. WSDL modelliert und beschreibt modulare Web Services und wird verwendet zur Dokumentation von verteilten Systemen und zur Automatisierung der Kommunikation zwischen Anwendungen. Folgende Arbeitsentw�rfe wurden in dem Zusammenhang aktualisiert und ver�ffentlicht: WSDL Additional MEPs, RDF Mapping und SOAP 1.1 Binding.