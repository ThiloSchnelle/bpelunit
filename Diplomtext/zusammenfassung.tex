\thispagestyle{empty} 
\vspace*{5cm}
\begin{center}
\textbf{Zusammenfassung}
\end{center}
Business Process Execution Language (BPEL) hat sich als ein Quasi-Standard f�r die Komposition mehrerer Web Services zu einem Gesch�ftsprozess etabliert. Trotz der breiten Akzeptanz dieser Sprache existieren noch sehr wenige Werkzeuge, die das Testen von BPEL-Prozessen unterst�tzen. Mit \textit{BPELUnit} wurde ein Framework entwickelt, der das strukturierte Testen einer BPEL-Komposition in Isolation (unabh�ngig von zusammengestellten Diensten) erm�glicht. Die Testabdeckungsanalyse, als Mittel zur Qualit�tskontrolle von Tests, wird durch das Framework nicht durchgef�hrt. Genau an dieser Stelle kn�pft diese Masterarbeit an.  

Es wird in der Arbeit untersucht, inwiefern die bekannten Testabdeckungsmetriken, die in konventionellen Programmiersprachen zur Bewertung der Tests eingesetzt werden, auf die BPEL-Sprache �bertragen werden k�nnen. Au�erdem werden neue f�r BPEL sinnvolle Metriken ausgearbeitet. Anschlie�end soll ein Konzept f�r die Integration dieser Metriken in das BPELUnit-Framework erarbeitet und implementiert werden.



BPEL Prozesse werden f�r die Verbindung der WEb Services zu einer Komposition verwendet. Diese Kompositionen spiegeln die Gesch�ftsprozesse die sehr wichtig f�r die Unternehmen sind. Daraus folgend die Korrektheit und die Robustheit des BPEL-Prozesses ist sehr wichtig. Das Testen von BPEL-Prozessen ist deswegen zwingend erforderlich. Aber es gibt keine praktischen Metriken, die verwendet werden k�nnen um die Testqualit�t oder Testvortschritt zu messen. In dieser Arbeit werden Testmetriken definiert wie Statementabdeckung und Zweig. Zus�tzlich werden BPEL-spezifischen Metriken eingef�hrt , die auf die wichtigen Eigenschaften der Sprache gerichtet sind. Anschlie�end wird die Erweiterung der BPELUnit Frameworks pr�sentiert. Dieses Unterst�tzt einfache Sammlung und Pr�sentation der Metriken. Dadurch k�nnen die Tester besser den Fortschritt und die Qualit�t ihrer Arbeit kontrollieren.  